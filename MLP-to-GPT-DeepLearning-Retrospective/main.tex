%! TeX program = pdflatex
\documentclass[10pt]{beamer}
%% ! TeX program = lualatex

%% Based on the original theme by Matthias Vogelgesang

\usetheme[progressbar=frametitle]{metropolis}
\usepackage{appendixnumberbeamer}

\usepackage{booktabs}
\usepackage[scale=2]{ccicons}

\usepackage{pgfplots}
\usepgfplotslibrary{dateplot}

\usepackage{xspace}
\newcommand{\themename}{\textbf{\textsc{metropolis}}\xspace}

\usepackage{graphicx} %LaTeX package to import graphics
\graphicspath{{images/}} %configuring the graphicx package

\usepackage{animate}

\usepackage{caption}
\captionsetup{font=small}

%%%%%%%%%%%%%%%%%%%%%%%%%%%%
%% UNCC Theme Adjustments %%
%%%%%%%%%%%%%%%%%%%%%%%%%%%%
\definecolor{CanvasBG}{HTML}{FAFAFA}

% From the official style guide
\definecolor{UnccGreen}{HTML}{005035}
\definecolor{UnccLightGreen}{HTML}{C3D7A4}
\definecolor{UnccGold}{HTML}{A49665}
\definecolor{UnccOrange}{HTML}{F3901D}
\definecolor{UnccLightYellow}{HTML}{899064}
\definecolor{UnccBlue}{HTML}{007377}
\definecolor{UnccPink}{HTML}{DE3A6E}
\definecolor{White}{HTML}{FFFFFF}
\definecolor{LightGray}{HTML}{F1E6B2}

\setbeamercolor{frametitle}{bg=UnccGreen}
\setbeamercolor{progress bar}{bg=UnccGold, fg=UnccGreen}
\setbeamercolor{alerted text}{fg=UnccOrange}

\setbeamercolor{block title}{bg=UnccGreen, fg=White}
\setbeamercolor{block title example}{bg=UnccBlue, fg=White}
\setbeamercolor{block title alerted}{bg=UnccPink, fg=White}
\setbeamercolor{block body}{bg=LightGray}

\metroset{titleformat=smallcaps, progressbar=foot}

\makeatletter
\setlength{\metropolis@progressinheadfoot@linewidth}{2pt}
\setlength{\metropolis@titleseparator@linewidth}{2pt}
\setlength{\metropolis@progressonsectionpage@linewidth}{2pt}

%%%%%%%%%%%%%%%%%%%%%%%%%%%%
%% UNCC Theme Adjustments %%
%%%%%%%%%%%%%%%%%%%%%%%%%%%%

\title{MLP to GPT: A Deep Learning Retrospective}
\subtitle{a brief history of break-throughs in nlp}

\begin{document}

\maketitle

% %=======================================================================================================================
% \begin{frame}[fragile]{What we plan to cover}
%
% 	- Make this the last slide you prepare
%
% 	% - Idea is\\
% 	% Goal (aka why should I care)\\
% 	% - If you are reading a research paper or referencing PyTorch/TF reference (you should be able to connect it to something)\\
% 	% - What's backprop\\
% 	% - linear regression\\
% 	% - MLP\\
% 	% - what's non-linearity\\
% 	% - main mathematical equation\\
% 	% - what are usages\\
% 	% - special cases\\
% 	% - RNN, CNN, GRU, LSTM\\
% 	% - Word Embeddings\\
% 	% - Language Models\\
% 	% - seq-to-seq models (explain the concept of time)\\
% 	% - explain how this create a time boundation (the issue solved by transformer later)\\
% 	% - What's (general or cross) attention?\\
% 	% - What's self attention\\
% 	% - Multi-headed attention\\
% 	%
% 	% - Transformers\\
% 	% - BERT\\
% 	% - GPT\\
% 	% - LLAMA\\
% 	% - MISTRAL\\
% 	% - Gemini / Gemma\\
% 	% - BARD\\
% 	% - Orca\\
% 	% - MPT-7B\\
% 	% - T5
% 	% - DeciLM-7B
% 	%
% 	% % any other relevant state-of-the-art models\\
% 	% - Topics for future / Things to look out for in DL space :\\
% 	% - RAG\\
% 	% % Look up recent research (+ chatGPT, Vishruit, Roshin etc.)
% 	%
% 	% Why should I care:
%
% \end{frame}

%=======================================================================================================================
\begin{frame}{Table of contents}
	\setbeamertemplate{section in toc}[sections numbered]
	\tableofcontents%[hideallsubsections]
\end{frame}

%#######################################################################################################################
\section[Pre-Transformer Developments]{Pre-Transformer Developments}

%=======================================================================================================================
\begin{frame}[fragile]{ Multi-layered Perceptron (1960s)}

	A \textbf{Multi-Layer Perceptron (MLP)} is a name for a modern feedforward artificial neural network
	\textit{consisting of fully connected neurons with a nonlinear activation function, organized in at least three
		layers}. Modern feedforward networks are trained using the \textbf{backpropagation method} and are
	colloquially referred to as the \textit{\textbf{"vanilla"}} neural networks.

	\begin{figure}[h]
		\centering
		\includegraphics[width=0.75\textwidth]{mlp_main}
		\caption{
			\href{https://playground.tensorflow.org/}{Tensorflow Playground}
		}
	\end{figure}

\end{frame}

%=======================================================================================================================
\begin{frame}[fragile]{Recurrent Neural Networks (1980s)}

	\textbf{Recurrent Neural Networks (RNNs)} are a class of neural networks designed for processing sequential
	data. The key differentiator with RNNs is the \textit{hidden state} which capture information (context) about
	the sequence so far, enabling them to "remember" previous inputs.
	\begin{equation}
		h_t = \tanh(W_h * h_{t-1} + W_x * x_t + b)
	\end{equation}

	% ### Key Features of RNNs:

	% **Sequential Data Handling:** RNNs can take sequences of variable length as input, making them suitable for
	% tasks like language modeling and machine translation.

	% - **Context Preservation**: RNNs maintain a hidden state that captures information from previous time steps, allowing
	% them to remember previous words and their contexts when processing current words. This capability is essential for
	% understanding language, where context can drastically change the meaning of a sentence.

	% **Shared Parameters:** The same parameters (weights) are used across all steps of the sequence, which reduces
	% the total number of parameters compared to fully connected networks.

	% - **Flexibility with Sequence Lengths**: RNNs can process sequences of varying lengths without requiring the input data
	% to be padded or truncated to a fixed size. This is important for NLP, where sentences and documents can vary greatly
	% in length.

	% ### Challenges:

	% 1. **Vanishing and Exploding Gradients:** During training, the gradients of the loss function can become very small (vanish) or very large (explode), making it difficult to learn long-range dependencies.

	% 2. **Training Complexity:** Due to their recurrent nature, RNNs can be more challenging to train than feedforward neural networks.

	% ### Applications:

	% 1. **Natural Language Processing:**
	% - Language modeling, text generation, machine translation, and sentiment analysis.

	% 2. **Time Series Prediction:**
	% - Stock price forecasting, weather prediction, and anomaly detection in sequences.

	% 3. **Speech Recognition:**
	% - Converting spoken language into text.

	% ### Example Use Case: Language Modeling

	% In language modeling, RNNs are used to predict the next word in a sentence. Given a sequence of words, the RNN
	% learns to model the probability distribution of the next word. This capability is essential for applications
	% like predictive text input and machine translation.

	\begin{figure}[h]
		\centering
		% TODO: Update this!
		% \animategraphics[autoplay,loop,width=0.60\textwidth]{35}{images/rnn_1/rnn_1-}{0}{35}
		\animategraphics[autoplay,loop,width=0.75\textwidth]{30}{images/rnn_1/rnn_1-}{0}{408}
		\caption{jalammar.github.io}
	\end{figure}

\end{frame}

%=======================================================================================================================
\begin{frame}[fragile]{Vanishing and Exploding Gradients}

	The vanishing and exploding gradient problems are common issues encountered during the training of recurrent
	neural networks (RNNs) due to large number of steps/layers in the backpropagation process wherein the gradients
	either decrease or increase exponentially in which case they become unusable (vanishing gradients carry too
	little information; exploding gradients are too unstable).

	\begin{figure}[h]
		\centering
		\includegraphics[width=0.75\textwidth]{vanishing_gradients}
		% \caption{
		% 	\href{https://playground.tensorflow.org/}{Tensorflow Playground}
		% }
	\end{figure}

	% ### Mitigating Strategies:

	% 1. **Weight Initialization:** Using appropriate initialization methods for network weights, such as Xavier or He
	% initialization, can help alleviate both vanishing and exploding gradient problems by ensuring that gradients
	% neither vanish nor explode as they propagate through the network.
	%
	% 2. **Activation Functions:** Choosing activation functions that avoid saturation (e.g., ReLU, Leaky ReLU) can
	% help mitigate the vanishing gradient problem by allowing gradients to propagate more effectively through the
	% network.
	%
	% 3. **Gradient Clipping:** Limiting the magnitude of gradients during training (gradient clipping) can prevent
	% the exploding gradient problem by ensuring that gradients do not become too large.
	%
	% 4. **Normalization Techniques:** Batch normalization and layer normalization can help stabilize training and
	% mitigate both vanishing and exploding gradient problems by normalizing the input to each layer.
	%
	% 5. **Architectural Modifications:** Using architectures like Long Short-Term Memory (LSTM) or Gated Recurrent
	% Unit (GRU) in RNNs can help mitigate the vanishing gradient problem by introducing mechanisms to better capture
	% long-range dependencies.
	%
	% By applying these strategies, practitioners can address the vanishing and exploding gradient problems, allowing
	% for more stable and effective training of deep neural networks.

\end{frame}

%=======================================================================================================================
\begin{frame}[fragile]{Long Short-Term Memory (LSTM)}

	\textit{It’s difficult for vanilla RNNs to learn to preserve information over many timesteps. How about an RNN with
		memory?}

	Long Short-Term Memory (LSTM) is a type of RNN architecture designed to address the traditional RNN's difficulty
	in learning long-term dependencies due to the vanishing gradient problem.

	\begin{figure}[h]
		\centering
		\includegraphics[width=0.75\textwidth]{LSTM}
		\caption{CS224N : NLP with Deep Learning}
	\end{figure}

\end{frame}


%=======================================================================================================================
\begin{frame}[fragile]{Long Short-Term Memory (LSTM)}

	LSTM has following key components:

	\begin{itemize}
		\item Memory Cell: Maintains information over time.
		\item Forget Gate: Decides what information to throw away from cell-state.
		\item Input Gate: Decides which new information to add to the cell-state.
		\item Output Gate: Controls how much of the cell state is exposed to the output.
	\end{itemize}

	\begin{figure}[h]
		\centering
		\includegraphics[width=0.75\textwidth]{LSTM}
	\end{figure}

\end{frame}



%=======================================================================================================================
\begin{frame}[fragile]{Recurrent Neural Networks}

	\begin{figure}[h]
		\centering
		\includegraphics[width=0.6\textwidth]{rnn_types}
	\end{figure}

\end{frame}

%=======================================================================================================================
\begin{frame}[fragile]{Recurrent Neural Networks : Sequence-To-Sequence}

	RNNs are often used in applications where input and output are in sequence format e.g. machine translation, text
	prediction, speech recognition, stock price forecasting etc.

	\begin{figure}[h]
		\centering
		\includegraphics[width=0.75\textwidth]{seq2seq_explained}
		\caption{Sequence to Sequence Learning with Neural Networks by Ilya Sutskever, et al}
	\end{figure}

\end{frame}

%=======================================================================================================================
\begin{frame}[fragile]{Recurrent Neural Networks : Encoders \& Decoders}

	The encoder processes the input sequence and distills it into a fixed-length representation, often referred to
	as the \textbf{context vector}. This vector serves as a condensed summary of the input, capturing its essence
	for the decoder to interpret.

	% Challenges:
	%
	% When faced with lengthy and information-dense input sequences, the model’s ability to maintain relevance across
	% the entire decoding process can wane. Not all fragments of the input’s context are equally pertinent at every
	% juncture of text generation. For instance, in machine translation, the word ‘boy’ in the sentence ‘A boy is
	% eating the banana’ does not necessitate the full breadth of the sentence’s context for accurate translation.

	\begin{figure}[h]
		\centering
		\includegraphics[width=0.75\textwidth]{encoder_decoder}
		\caption{Seq2Seq Learning with Neural Networks by Ilya Sutskever, et al}
	\end{figure}

\end{frame}

%=======================================================================================================================
\begin{frame}[fragile]{Recurrent Neural Network : Encoders \& Decoders}

	\begin{figure}[h]
		\centering
		% TODO: Update this!
		% \animategraphics[autoplay,loop,width=0.75\textwidth]{10}{images/seq2seq_2/seq2seq_2-}{1}{10}
		\animategraphics[autoplay,loop,width=0.75\textwidth]{40}{images/seq2seq_2/seq2seq_2-}{1}{573}
		% \caption{
		% 	\href{https://playground.tensorflow.org/}{Tensorflow Playground}
		% }
	\end{figure}

	\begin{figure}[h]
		\centering
		% TODO: Update this!
		% \animategraphics[autoplay,loop,width=0.75\textwidth]{10}{images/seq2seq_3/seq2seq_3-}{1}{10}
		\animategraphics[autoplay,loop,width=0.75\textwidth]{35}{images/seq2seq_3/seq2seq_3-}{1}{519}
		\caption{jalammar.github.io}
	\end{figure}

\end{frame}

%#######################################################################################################################
\section[Transformers]{Transformers}

%=======================================================================================================================
\begin{frame}[fragile]{Attention}

	Enable the theme by loading

\end{frame}

%=======================================================================================================================
\begin{frame}[fragile]{Self-Attention}

	Enable the theme by loading

\end{frame}

%=======================================================================================================================
\begin{frame}[fragile]{multi-head attention}

	Enable the theme by loading

\end{frame}

%=======================================================================================================================
\begin{frame}[fragile]{Transformers}

	Transformers and attention mechanisms have revolutionized the field of deep learning, offering a powerful way to
	process sequential data and capture long-range dependencies.

	In this article, we will put our serious hat on and truly explore the basics of transformers and the importance of
	attention mechanisms in enhancing model performance and coherence.

	Key Takeaways
	- Attention mechanisms are crucial in transformers, allowing different tokens to be weighted based on their
	importance, enhancing model context and output quality.
	- Transformers operate on self-attention, enabling the capture of long-range dependencies without sequential
	processing.
	- Multi-head attention in transformers enhances model performance by allowing the model to focus on different
	aspects of the input data simultaneously.
	- Transformers outperform RNNs and LSTMs in handling sequential data due to their parallel processing
	capabilities.
	- Applications of transformers span across NLP, computer vision, and state-of-the-art model development.
	Evolution of Transformers in Deep Learning

	Long-Range Dependencies

	One of the most salient features of Transformers is their innate ability to handle long-range dependencies with
	aplomb. This characteristic is crucial for understanding the context in sequences where relevant information may be
	separated by considerable distances. Traditional Recurrent Neural Networks (RNNs), including their more advanced
	variants like Long Short-Term Memory (LSTM) networks, often grapple with the vanishing gradient problem. This issue
	causes the influence of initial inputs to wane as the sequence progresses, making it arduous for these models to
	maintain context over long sequences.

	Transformers, by contrast, are not constrained by sequence length. Their self-attention mechanism computes the
	relevance of all parts of the input sequence simultaneously, allowing for a global understanding of the data. This
	is a game-changer for tasks that require the synthesis of information spread across an entire sequence, such as
	document summarization or question answering.

	The following list delineates the comparative advantages of Transformers over RNNs in handling long-range
	dependencies:

	Transformers: Utilize self-attention to weigh the importance of each element in the sequence, regardless of
	distance. RNNs: Sequential processing can lead to diminished influence from earlier elements, especially in longer
	sequences. LSTMs: Incorporate gating mechanisms to better retain information over time, but still face challenges
	with very long sequences.

	In essence, the Transformer architecture has redefined the landscape of sequence modeling by providing a robust
	solution to the long-standing challenge of long-range dependencies. This has opened up new vistas in the field of
	machine learning, particularly in complex tasks that require deep contextual understanding.

\end{frame}

%#######################################################################################################################
\section[Transformer-Based Models]{Transformer-Based Models}

%=======================================================================================================================
\begin{frame}[fragile]{BERT}

	Enable the theme by loading

\end{frame}

%=======================================================================================================================
\begin{frame}[fragile]{GPT}

	Enable the theme by loading

\end{frame}

%=======================================================================================================================
\begin{frame}[fragile]{Llama}

	Enable the theme by loading

\end{frame}

%=======================================================================================================================
\begin{frame}[fragile]{Mistral}

	Enable the theme by loading

\end{frame}

%#######################################################################################################################
\section[Quickstart]{quickstart}

%=======================================================================================================================
\begin{frame}[fragile]{starting a new nlp project}

	Starting Point for a new project -\\
	- Task based split\\
	- General Papers\\

\end{frame}

%=======================================================================================================================
\begin{frame}[fragile]{starting a new DL project}

	Github Reference Guide

\end{frame}

%=======================================================================================================================
\begin{frame}[fragile]{resources}

	PapersWithCode\\
	MadeWithML

\end{frame}

%=======================================================================================================================
\begin{frame}[fragile]{resources}

	Awesome NLP Papers

\end{frame}

%=======================================================================================================================
\begin{frame}[fragile]{Brief Recent History of NLP Deep-Learning Models}
	\begin{itemize}

		\item \textbf{1965:} \textit{First feedforward network} was published by Alexey Grigorevich Ivakhnenko and Valentin Lapa
		      % Q: How was this network trained? 

		\item \textbf{1967:} \textit{Stochastic gradient descent} was used for the first time for training neural network

		\item \textbf{1982:} \textit{BackPropogation} method was applied in the way that has become standard, for the first time by Paul Werbos.

		\item \textbf{1986:} David Rumelhart, Geoffrey Hinton, and Ronald Williams developed the idea of
		      \textit{backpropagation through time (BPTT)} to train RNNs.

		      % This was a significant step forward because it provided a method to train RNNs by unfolding them
		      % in time and applying backpropagation.

		\item \textbf{1997:} Sepp Hochreiter and Jürgen Schmidhuber introduced \textit{Long Short-Term Memory (LSTM)} networks.

		      % LSTMs addressed the vanishing gradient problem inherent in traditional RNNs, making it feasible
		      % to learn long-term dependencies. LSTMs use memory cells and gating mechanisms to better capture
		      % long-range dependencies in sequence data.

	\end{itemize}
\end{frame}

\begin{frame}[fragile]{Brief Recent History of NLP Deep-Learning Models}
	\begin{itemize}

		\item \textbf{2013:} Mikolov et al. developed \textit{Word2Vec} to produce word embeddings (dense vector
		      representations of words in a continuous vector space).

		      % Word2Vec is a set of neural network models designed to produce word embeddings, which are dense
		      % vector representations of words in a continuous vector space. These embeddings capture semantic
		      % relationships between words based on their context in a large corpus of text.

		\item \textbf{2014:} Ilya Sutskever et al. introduced the \textit{Sequence-to-Sequence (Seq2Seq)} model,
		      a type of RNN architecture which used \textit{encoder-decoder framework} for tasks such as
		      machine translation.

		      % - **Cho et al.** introduced the Gated Recurrent Unit (GRU), a simpler alternative to LSTMs that also addresses the
		      % vanishing gradient problem and improves training efficiency.

		\item \textbf{2015-2016:} \textit{Attention Mechanisms} were introduced to improve Seq2Seq models.

		      % Attention allowed models to focus on different parts of the input sequence dynamically, greatly
		      % enhancing the performance of RNNs in tasks like machine translation.

		\item \textbf{2017:} Vaswani et al. published the \textit{"Attention Is All You Need"} paper, introducing
		      the Transformer model.

		      % Transformers, which rely entirely on self-attention mechanisms and dispense
		      % with recurrence, began to replace RNNs and LSTMs in many sequence modeling tasks due to their
		      % efficiency and superior performance on long sequences.

		      % ### 2020s: RNNs in Context
		      %
		      % 11. **Present**:
		      % - While Transformers and their derivatives (e.g., BERT, GPT, T5) have become the dominant models for many sequence
		      % tasks, RNNs, LSTMs, and GRUs are still widely used, especially in applications where their specific properties are
		      % advantageous, such as certain types of time-series forecasting, speech recognition, and applications with limited
		      % computational resources.

	\end{itemize}
\end{frame}

%=======================================================================================================================
\begin{frame}{References}
	Some references to showcase [allowframebreaks] \cite{yang2023harnessing}
\end{frame}

% %#######################################################################################################################
% \section{Conclusion}

%=======================================================================================================================
{\setbeamercolor{palette primary}{fg=white, bg=UnccGreen}
\begin{frame}[standout]
	\textbf{Questions?}
\end{frame}
}

%=======================================================================================================================
\begin{frame}[allowframebreaks]{References}

	\bibliography{main}
	\bibliographystyle{abbrv}

\end{frame}

\end{document}
